\documentclass{beamer}
\usepackage[utf8]{inputenc}
\usetheme{Madrid}
\title{Economic growth and comparative development Replication paper}
\author{María Gutiérrez, Santiago Prieto, Nicolas Urrego}
\date{June 2020}
\begin{document}
\maketitle

\Title {The Interaction between the Quantity and Quality of Children}
\Author {By: Gary S. Becker and H. Gregg Lewis}

\begin{frame}
\tableofcontents[pausesections]
\alert{Table of contents}
\subsection{Summary}
\subsection{Introduction}
\subsection{Mathematics}
\subsection{Interpretations}
\subsection{Arguments} 
\subsection{Conclutions}
\end{frame}



\begin{frame}{Introduction}
\begin{itemize}
\item \alert{Other papers}:negative relation between quantity and quality often observed is a consequence of a low substitution elasticity in a family´s utility function between parent´s consumption or level of living and that of the children. (Duesenberry 1960; Willis 1969)


\item \alert{This paper}: different but it makes equally special assumptions about the substitution between \alert{quantity} and \alert{quality} in the utility function and in household production. (Becker, Lewis. p. S279)
\end{itemize}
\end{frame}


\begin{frame}{Summary}
Over the years, a demographic transition has occurred. This has represented the change in the size of the world's population as the time passes. In some countries it has increased, in others, on the contrary, it has decreased. The Malthusian theory, proposed by the Anglican clergyman Thomas Robert Malthus, explains the fact that over time, a phenomenon has occurred in different parts of the world in which there is an exponential progression of the population that also results in an increase in production. Now, in this theory a special emphasis is made on the fact that technology allows the change in the product to be greater than the change in the population consequent with a growth in the gross domestic product of a territory.This paper, which is a replica of the paper "On the Interaction between the Quantity and Quality of Children" written by Gary S. Becker and H. Gregg Lewis explores how this theory can be seen and analyzed in reality.
\end{frame}

\begin{frame}{Introduction}
\alert {Utility funtion}
$$U = U(n, q, y)$$ s.t $$I=nq\pi+ y\pi_y$$ 
Where n is the number of children, q the quality of the children (assumed to be the same for all the children), y the rate of consumption of all other commodities, I the total income, $\pi$ the price of nq and $\pi_y$ the price of y.
\end{frame}

\begin{frame}
\alert {First order conditions}
$$MU_n = \lambda q \pi = \lambda p_n$$ 
$$MU_q = \lambda n \pi = \lambda p_q$$
$$MU_y = \lambda p_y = \lambda p_y$$
MU's are the marginal utilities, p's are the marginal costs or shadow price of children with respect to number $p_n$ is positively related to q, the level of quality $(p_q)$ is positively related to n, the number of children.   
\end{frame}

\begin{frame}
\begin{itemize}
\item {An increase in quality is more expensive if there are more children.}
\item {An increase in quantity is more expensive if the children are of higher quality.}
\end{itemize}
The implication that this may have in price effects and income haven't been explored. 
\end{frame}

\begin{frame}
\alert{Income effects} 
Income elasticity's of demand for the number (n) of children that has a quality and for all other commodities (y) be $\Upsilon_n$, $\Upsilon_q$ and $\Upsilon_y$ respectively. This elasticity's are derived by the changes in income while holding prices constant. 
The appropriate prices for this purpose are the shadow prices (wich are the marginal rates of sustitution in the utility function) p_n, p_q, and p_y.
\end{frame}


\begin{frame}
For all of this \alert{Income} may be: 
$$R= np_n+qp_q+yp_y$$
$$R= I + nq\pi$$
Which is the total expenditure on the number of children, their quality and y, calculated at the shadow prices. 



The \alert{mean value} of the true \alert{income elasticities} is:
$$1=\frac{np_n}{R}\Upsilon_n+\frac{qp_q}{R}\Upsilon_q+\frac{yp_y}{R}\Upsilon_y$$
\end{frame}

\begin{frame}
The \alert{weighet mean} of the observed income elasticities is $$\frac{I}{R}=\frac{I}{I+nq\pi}$$
This is less than a unity: that is:
$$1>\frac{I}{R}=\frac{I}{I+nq\pi}$$
$$1>\frac{I}{R}=\frac{np_n}{R}\Upsilon_n+\frac{qp_q}{R}\Upsilon_q+\frac{yp_y}{R}\Upsilon_y+$$
Where all the $\Upsilon$'s are fixed.All of this means that on the average, the observed elasticities are smaller than the true elasticities on the ratio $frac{I}{R}$



If I increase, n, q and y increase.
If n and q increase, then the shadow prices $p_n$ and $p_q$ rise.
\end{frame}

\begin{frame}
\begin{enumerate}
    \item {Money income increase - wage rates increase = price effect}
    \item{In ratio terms the increase is less in real income than in money income because the cost of producing commodities in the household increase by the rise in the price.}
\end{enumerate}
\end{frame}

\begin{frame}
\begin{itemize}
    \item The true income elasticity with respect to quality $(\Upsilon_q)$ is larger than that with respect to quality.
    
    \item The observed elasticity for quantity $(\Upsilon_n)$ may be negative even though the true elasticity $(\Upsilon_n)$ is not.
    
    \item $\Upsilon_q$ declines as income I rises
\end{itemize}
\end{frame}

\begin{frame}
\alert{Price effects} 
The budget constraint is:
$$p_n=n\pi_n+nq\pi+q\pi_q+yp_y$$
So that the shadow prices or marginal costs are now
$$p_n=\pi_n+q\pi
$$p_n=\pi_q+n\pi$$
$$p_y=\pi_y$$

These shadow prices for n and q each contain a "fixed" component: $\pi$.

The component $\pi$ in child costs consist of costs that depend on quantity but not in quality.

Examples: Prenatal child costs, such as maternity care.
\end{frame}

\begin{frame}
\begin{itemize}
    \item A reduction in the shadow prices of quality $$p_q=pi_q+n\pi$$. This induces a substitution in favor of quantity.
    \item This results in a \alert{fall of quantity} and an \alert{increase in quality}.
    \item Also, an increase in the education of parents $$\pi_q$$ make a \alert{fall of quantity} and an \alert{increase in quality}. This is also an increase in the shadow price of quantity $$(p_n=\pi_n+q_\pi)$$
\end{itemize}
\end{frame}

\begin{frame}
\begin{enumerate}
    \item Quantity and quality are closely related  because the shadow price of quality depends on quantity and the shadow price of quantity depends on quality.
    \item The income compensated elasticity of quantity with respect to equal percentage changes in $$\pi_n, \pi_q and\pi$$ tends to be greater than the elasticity for quality.
    \item \alert{Example}: An increase in women's  wage rates reduces the number of children by a much bigger percentage than the quality of children. 
\end{enumerate}
\end{frame}

\begin{frame}
\begin{itemize}
    \item The observed price elasticity of quantity exceeds that of quality, just the opposite of our conclusion for observed income elasticity's.
    \item This reversal is not only unexpected, but it also gives a consistent interpretation to literature existence.  
\end{itemize}
\end{frame}


\end{document}

